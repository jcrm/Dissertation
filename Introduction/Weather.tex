\subsection{Weather:}
Weather has been a major part of games since the early 1980’s.
There are two types of weather games, those where the weather affects gameplay and those where it adds a depth of emersion.
Looking deeper into weather and how it is used in games, clouds stand out as the most used part of the weather in games.
It has been used in games such as Ouranos! (\citeyear{Ouranos80}) and Cloud (\citeyear{Cloud05}) which use clouds as a major part in how the game is played.
Whereas games such as Pole Position (\citeyear{PolePosition82}), Super Mario Bros (\citeyear{SMB85}) and Tomb Raider (\citeyear{TombRaider13}) use clouds as immersion devices.
There are some games such as Microsoft Flight Simulator X (\citeyear{MFS03}) and NCAA Football 14 (\citeyear{NCAAF13}) which use clouds as part of gameplay and an immersion tool. 

All the previous example create clouds differently Ouranos! (\citeyear{Ouranos80}) uses ASCII to create and render clouds mainly due to graphical limitations at the time.
However Pole Position (\citeyear{PolePosition82}) and Super Mario Bros (\citeyear{SMB85}) use 2D sprites to render clouds to the background while Tomb Raider (\citeyear{TombRaider13}) uses 3D scripted clouds.
Both Microsoft Flight Simulator X (\citeyear{MFS03}) and NCAA Football 14 (\citeyear{NCAAF13}) can use location data to simulate weather when playing the game as well as having dynamic weather systems.
The majority of games including those mentioned above only use artistic representations of clouds instead of creating them in real time using equations.