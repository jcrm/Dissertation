\subsection{Research Question:}
\label{sec:rq}
The project aims to create realistically moving and looking clouds in real time using fluid dynamic equations.
Another aim is to create rain in locations and an appropriate amount that relates to the clouds created.
The last aim is to compute the equations in the GPU so that the equations can be computed efficiently.
These aims lead to the research question:

\textit{\textbf{How can the amount of Precipitation generated in a game be related to realistic simulated clouds generated in real-time using fluid dynamic equations?}}

Answering the research question results in a number of objectives the need to be completed including using a number of equations, mainly fluid dynamic equations in 3D, space to generate and move clouds.
To use these clouds to generate the rain in a 3D scene, the fluid dynamic equations will need to be optimized to run as smoothly as possible in the application whilst producing realistic, or at least plausible, effects..
The final application will be analysed visually and numerically to test how the clouds grow and move over time, as well as generating rain in the most efficient way.