\subsubsection{Fluid Dynamics:}
\label{sec:fd}
As clouds can be described as an incompressible fluid it can be simulated via the fluid dynamic equations.
The Navier-Stokes Equations are used for a “fluid that conserves both mass and momentum.” \citep{JStam99}.
In the Navier-Stokes equations $\rho$ is the density, $\mathbf{f}$ represents all external forces and $\nu$ is the kinematic viscosity of the fluid.
The velocity and pressure are defined as $\mathbf{u}$ and $p$ respectively.
The second equation is the continuity equation which means the fluid is incompressible.
\begin{equation} \label{eq:Navier-Stokes}
  \centering
  \frac{\partial \mathbf{u}}{\partial t}=-\left(\mathbf{u}\cdot \nabla \right)\mathbf{u}-\frac{\nabla p}{\rho}+\nu{\nabla }^{2}\mathbf{u}+\mathbf{f}
\end{equation}
\begin{equation} \label{eq:Continuity Equation}
  \centering
  \nabla\cdot\mathbf{u}=0
\end{equation}
The Navier-Stokes equations (\ref{eq:Navier-Stokes}) and (\ref{eq:Continuity Equation}) can be simplified to Euler's Equation when “the effects of viscosity are negligible in gases” \citep*{Fedkiw01}.
This makes the equations for generating the clouds less computationally heavy and can be shown in equation (\ref{eq:Euler's Equation}) which has no $\nu{\nabla }^{2}\mathbf{u}$.
The continuity equation has not changed and can be seen from (\ref{eq:Continuity Equation Euler}) being the same as (\ref{eq:Continuity Equation}).
\begin{equation} \label{eq:Euler's Equation}
  \centering
   \frac{\partial \mathbf{u}}{\partial t}=-\left(\mathbf{u}\cdot \nabla \right)\mathbf{u}-\frac{\nabla p}{\rho}+\mathbf{f}
\end{equation}
\begin{equation} \label{eq:Continuity Equation Euler}
  \centering
  \nabla\cdot\mathbf{u}=0
\end{equation}
\citet*{Fedkiw01}, \citet{HarrisEtAl03}, and \citet*{DOverby02} all used work created by \citet{JStam99} on stable fluid simulations.
\citet*{DOverby02} used the actual solver created by \citet{JStam99} in the application to solve the fluid dynamic part of creating clouds.
Whereas \citet*{Fedkiw01} and \citet{HarrisEtAl03} used the theory in the creation of the smoke and clouds respectively.
Even though all three used the same start for simulating cloud generation they have different methods for assigning values to the other equations needed. 

The \citet*{Fedkiw01} model uses a Poisson equation to compute the pressure of the system and two scalar functions for advecting the temperature and density.
This model also uses a function built up of the temperature, ambient temperature, density, and two other positive constants to create a buoyancy effect.
The model also simulates velocity fields, which are dampened out on the coarse grid, by finding where the feature should be and then creating a realistic turbulent effect.

\citet*{DOverby02} computes the local temperature based upon the heat energy and the pressure.
The pressure is calculated from ground level to the tropopause by an exponentially decreasing value \citep*{DOverby02}.
The buoyancy in this model is created using the local temperature, the surrounding temperature and a buoyancy scalar.
Relative humidity is calculated based upon current water vapour and saturated water vapour.
Water condensation is then calculated based upon relative humidity, hygroscopic nuclei, and a condensation constants.
The final equation to be calculated is the latent heat which is calculated by the water condensation and a constant.
 
The \citet{HarrisEtAl03} model uses equations for water continuity, thermodynamics, buoyancy, and a Poisson equation for fluid flow.
This model also creates velocity fields using the same process as \citet*{Fedkiw01}.
This model uses more complicated equations than the previous two models to more accurately simulate the creation of clouds.
For example this model uses gravity, the mass mixing ratio of hydrometeors and virtual potential temperatures, whereas the previous models use scalars or other constants with the temperature to create the buoyancy force.

The vorticity confinement as defined by \citet{HarrisEtAl03} model is defined in equation (\ref{eq:vc}).
$\boldsymbol{\omega}$ is the vorticity, defined by $\boldsymbol{\omega} = \nabla\times\mathbf{u}$, and $\mathbf{N}$ is the normalized vorticity vector field and points from areas of lower vorticity to areas of higher vorticity which is defined by equation (\ref{eq:vcn}).
With $h$ is the grid scale and $\epsilon$ is a scale parameter.

\begin{equation} \label{eq:vc}
  \centering
  \mathbf{f}_{vc} = \epsilon h\left(\mathbf{N}\times\boldsymbol{\omega}\right)
\end{equation}
\begin{equation} \label{eq:vcn}
  \centering
  \mathbf{N} = \frac{\nabla\left|\boldsymbol{\omega}\right|}{\left|\nabla\left|\boldsymbol{\omega}\right|\right|}
\end{equation}

The buoyant force is defined in equation (\ref{eq:bf}) where $g$ is the acceleration due to gravity.
$q_{v}$ is the mixing ratio of hydrometeors and in this case is defined as $q_{c}$, the mixing ratio of liquid water.
$\theta_{v}$ is the virtual potential temperature and is defined in equation $\theta_{v} \approx \theta(1+0.61 q_{v})$.
$\theta_{v0}$ is the reference potential temperature and is between 290 and 300K as defined by \citet{HarrisEtAl03}.

\begin{equation} \label{eq:bf}
  \centering
  B = g\left(\frac{\theta_{v}}{\theta_{v0}} - q_{h}\right)
\end{equation}

The water continuity in \citet{HarrisEtAl03} model is based upon the Bulk Water Continuity model which described by \citet{houze1994cloud} as “the simplest type of cloud is a warm non-precipitating cloud”.
\citet{houze1994cloud} describes the model as a set of categories in which clouds can be created.
The categories for this simple type of cloud model are vapour $q_{v}$ and cloud liquid water $q_{c}$ and are described in equations (\ref{eq:wc}) by \citet{houze1994cloud}.
$C$ is the condensation rate.

\begin{equation} \label{eq:wc}
  \centering
  \frac{Dq_{v}}{Dt} = - C,	\frac{Dq_{c}}{Dt} = C
\end{equation}

The thermodynamic equation defined in \citet{HarrisEtAl03} model can be seen in equation (\ref{eq:thermo}). $c_{p}$ is the specific heat capacity of dry air at constant pressure in this case $1005J kg^{-1} K^{-1}$. $L$ is the latent heat of vaporization of water which is $2.501 J kg^{-1}$.
The part of the equation in brackets on the right hand side of the equation can be exchanged for the condensation rate from the water continuity equations.

\begin{equation} \label{eq:thermo}
  \centering
  \frac{\partial\theta}{\partial t} + \left(\mathbf{u}\cdot\nabla\right)\theta = \frac{-L}{c_{p}\Pi}\left(\frac{\partial q_{v}}{\partial t} + \left(\mathbf{u\cdot\nabla}q_{v}\right)\right)
\end{equation}

$\Pi$ is called the Exner function and is defined in equation \ref{eq:exner}.
Where $p_{0}$ is the pressure at the surface, usually taken as $1000 hPa$; $R_{d}$ is the gas constant for dry air a can be taken as $287 J kg^{-1} K^{-1}$; $c_{p}$ is the heat capacity of dry air at constant pressure, and $p$ is pressure.

\begin{equation} \label{eq:exner}
  \centering
  \Pi = \left(\frac{p}{p_{0}}\right)^{\frac{R_{d}}{c_{p}}}
\end{equation}

With these extra equations using fluid dynamics for generating and moving clouds will give more accurate simulations compared to the previously mentioned processes. There is a draw back for using fluid dynamic equations as it will use more computing power compared to artistically created, and Cellular Automaton methods. 