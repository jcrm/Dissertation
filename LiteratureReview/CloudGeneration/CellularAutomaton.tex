\subsubsection{Cellular Automata (CA):}
\label{sec:ca}
A Cellular Automaton can be described as a regular shaped structure which consists of identical cells that are computed synchronously depending on the state of the cell and its neighbours \citep{SDantchev11}.
\citet{DobashiEtAl00} used a cellular automaton model when generating the clouds which involved giving each cell a number of boolean states that, when coupled with the rules generated clouds.

This method was extended by \citet{Miyazaki01} who used the Coupled Map Lattice (CML) method which is described as “an extension of cellular automaton, and the simulation space is subdivided into lattices”.
\citet{Miyazaki01} also goes on to explain that the CML model differs from the CA model by using continuous values instead of discreet values.
This CML model uses very simple equations for viscosity and pressure effects, advection, diffusion of water vapour, thermal diffusion and buoyancy, and the transition from vapour to water.

Cellular Automaton gives a lot more control to the physics of clouds because of the equations used to define them compared to clouds created by artists.
However these equations are not as accurate as using fluid dynamic equations to move and generate clouds.