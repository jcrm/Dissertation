\subsection{Rain Rendering:}
\label{sec:rain}
“Rain is an extremely complex natural atmospheric phenomenon” \citep*{APuig-Centelles09}.
\citet*{APuig-Centelles09} describe two main techniques for rendering rain to a scene scrolling textures where a texture the size of the screen scrolls in the direction of the rain, and a particle system where each rain drop is represented as a particle in the system.
\citet{STariq07} writes “animating rain using a particle system is more useful for realistic looking rain with lots of behaviour (like changing wind).”
\citet*{APuig-Centelles09} sates that texture scrolling “is faster than particle systems, but it does not allow interaction between rain and the environment.”

An extension to the Bulk Water Continuity model which was described in section \ref{sec:fd} allows a warm precipitating cloud with rain as an additional category, \citet{houze1994cloud}.
Now instead of two equations there are three which are shown in equations \ref{eq:dqv} - \ref{eq:dqr}.

\begin{equation} \label{eq:dqv}
  \centering
  \frac{dq_{v}}{dt} = - C + E_{c} + E_{r}
\end{equation}
\begin{equation} \label{eq:dqc}
  \centering
  \frac{dq_{c}}{dt} = C - A - K - E_{c}
\end{equation}
\begin{equation} \label{eq:dqr}
  \centering
  \frac{dq_{r}}{dt} = A + K + F - E_{r}
\end{equation}

In the above three equations $C$ represents the condensation of vapour into cloud water.
$A$ is the autoconversion which is the rate cloud water decreases as particles grow to raining size.
$E_{c}$ and $E_{r}$  are evaporation variables, the former of cloud water and the latter is evaporation of rainwater.
$K$ is the collection of cloud water and $F$ represents the rain fallout of the model.
Adding the third equation and the extra variables to the water continuity from section \ref{sec:fd} will allow for more realistic rain generation with a particle system.
