\subsection{Resource Usage:}
\label{sec:resource}

Table \ref{tbl:grid} shows the application using a number of different grid sizes for the textures used in the application.
For a number of these different grid size the amount of Jacobi Solver iterations is slightly different to see how changing this impacts the efficiency of the application.
The values for the different iterations are 16, 20, and 32 and they are used for the $32*32*32$ and $64*64*64$ grids.

\begin{table}
\centering
\centering\caption{Texture Grid Size Performance Analysis}
\label{tbl:grid}
    \begin{tabular}{c|cccc}
    Grid Dimensions\\(Width * Height * Depth) & Jacobi Iterations & FPS(avg) & GPU (\%) & CPU (\%) \\ \hline\hline
    32*32*32                                  & 16                 & 60         & 25.8                  & 46.6                  \\
    32*32*32                                  & 20                 & 57         & 32.2                  & 57.0                  \\
    32*32*32                                  & 32                 & 50         & 32.3                  & 55.4                  \\
    64*64*64                                  & 16                 & 30         & 41.6                  & 58.3                  \\
    64*64*64                                  & 20                 & 26         & 45.9                  & 64.4                  \\
    64*64*64                                  & 32                 & 18         & 51.0                  & 61.9                  \\
    128*128*128                               & 16                 & 5          & NA                    & 54.8                  \\
    64*64*32                                  & 16                 & 38         & 37.1                  & 51.2                  \\
    128*128*32                                & 16                 & 16         & 31.6                  & 39.6                  \\
    128*128*64                                & 16                 & 10         & 29.8                  & 32.5                  \\
    \end{tabular}
\end{table}

With the different grid sizes Table \ref{tbl:grid} shows how many frames per second (FPS) the application runs at.
This was calculated by running the application a number of different times and averaging the FPS.
The GPU and CPU columns are values taken from the Nsight Visual Studio Edition performance analyser.
These values represent the utilization of either the GPU or CPU while the application is running.
These values may not be exact to when the application is running due to Nsight performance analyser using resources that wouldn't normally be used.

\begin{table}[h]
\centering
\centering\caption{CUDA Kernel Resource Usage}
\label{tbl:ck}
    \begin{tabular}{c|cc}
    \bf{CUDA Kernels} & \bf{Duration} (avg $\mu s$) & \bf{Occupancy} (\%) \\ \hline\hline
    Advect Velocity        & 3835                         & 62.5           \\
    Advect Temperature     & 2079                         & 100            \\
    Vorticity Confinement  & 1284                         & 100            \\
    Forces				   & 1816                         & 100            \\
    Buoyancy Force         & 3222                         & 100            \\
    Water Continuity       & 10125                        & 75             \\
    Update Temperature     & 2584                         & 100            \\
    Divergence             & 839                          & 100            \\
    Jacobi Solver          & 18490                        & 100            \\
    Cloud Velocity Update  & 1577                         & 100            \\
    Flatten Rain  		   & 223                          & 100            \\
    \end{tabular}
\end{table}

Table \ref{tbl:ck} shows the time on average each kernel takes every frame whilst the application is running.
The time take is in $\mu s$ and is taken from the Nsight performance analyser using a texture size of $64*64*64$ and 20 Jacobi Solver iterations.
The Occupancy column shows how well each kernel uses the resources of the device.
NVIDIA (\citeyear{nvsns2012}) states that “higher occupancy does not always equate to higher performance … low occupancy always interferes with the ability to hide memory latency, resulting in performance degradation.”
There are some kernels not in the table such as the initialise kernels, as they do not affect the running of the application at each frame.
