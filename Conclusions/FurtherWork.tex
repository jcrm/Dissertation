\subsection{Further Work:}
\label{sec:fw}
To start looking at where the project could lead the most obvious answer is for the use of more realistic cloud and rain systems in games and in turn more physically accurate simulations of weather.
This prospect will take some work and most likely won't be available for a long time due to the constraints the hardware provides on when calculating the properties of the cloud system.
However with better optimization and more efficient graphic cards the grid size the cloud system uses can increase and in turn a larger area can be computed on the same size are can be computed more accurately.
As mentioned in section \ref{sec:disscus} changing from the simple Jacobi Solver to a more complex but faster solver such as Multi-Grid could lead to better optimization of the system and in return more resources for the cloud system to use.

Another improvement to the cloud system which doesn't involve the increase of hardware for the system to run on only the optimization of the current system.
This improvement would be to use the Navier-Stokes equation instead of the less detailed Euler's Equation which both are described in section %%.
Using the Navier-Stokes equations will give make the physical components of the cloud system more realistic by how much it is difficult to know.
However it the current system would require a large number of changes and optimization to get to a code base in which the Navier-Stokes equations can run efficiently. 
As well as more accurate physical simulation of cloud the clouds in this system could visually be increased via the use of some of the techniques mentioned in the literature review in section \ref{sec:cr}.
The techniques such as single scattering or multiple scattering will greatly improve how the cloud looks. 

Clouds are not the only area of the system that could be improved or extended the rain in the system could be rendered using more realistic techniques instead of just using a plain texture and billboarding.
The rain similar to the cloud could be rendered with light scattering effects the only problem lies in efficiency of the rendering techniques. 
The above techniques for making the clouds and rain more realistic either physically or visually are not the only way the application could be improved or extended.
As described in section \ref{sec:rain} the Bulk Water model uses mixing ratio of water vapour, cloud water, and rain to calculate when to generate rain.
\cite{houze1994cloud} state that this model can be extended from a warm precipitating cloud with rain that this system uses to a cloud in which snow can be created.
To do this an extra mixing ratio need to be added to the current Bulk Water model being used, and small variations made to the equations being solved for water continuity.

The long term goal for this topic would be the ability to create atmosphere weather system similarly to those used to predict the weather but a lot less complex and on a much smaller scale.
This would be more useful for certain games than other such as open-world games, or massive multiple online role playing games, as the scale of these worlds would mean that weather would change whilst moving around the game world, as well as making the game experience different for each player.
This type of physically realistic and visually accurate system is a long way off as it would use a large amount of resources which can be used on other parts of the game.