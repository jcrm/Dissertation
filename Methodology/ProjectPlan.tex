\subsection{Project Plan:}
To start the project a basic two dimensional application will be created that will test the complexity needed for the other functions being used in this model such as the buoyancy equation.
This model will also help when figuring out the moment to generate rain.
After this application is created a basic 3D application will be created which will consist of an uneven terrain and a camera. 
The next part of the project to be undertaken will be turning the 2D simulation code into 3D simulation code.
After there is code for creating a 3D simulation of the clouds the next step will be to render these clouds in 3D.
Next generating the rain will be added to the application.
This step will allow for the created and destruction of the rain at the right moments of the clouds life.
There are two more steps related to the application in the project plan.
The first being optimisation and the second being evaluation.
There will be some overlap between these two stages as the optimisation could lead to an increase in application efficiency and will be part of the evaluation methodology. 

There are two other stage that will be carried out whilst the application is being created and tested.
These are writing the draft dissertation and the final dissertation.
The reason these will be written during the creation of the application and not after testing is to give enough time to write the dissertation to a high standard.

This timetable can be seen in Gantt Chart form in the Appendix as Figure 6.5. 